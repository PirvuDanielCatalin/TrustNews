\hfill

Noțiunea de știri false (\textit{fake news}) a devenit de câțiva ani un fenomen din ce în ce mai răspândit pe toate mediile de comunicare. Efectele periculoase ale ei se întind de la o simplă preluare a unei informații greșite mai mult sau mai puțin voite până la adevărate campanii de dezinformare sau propagandă în favoarea cuiva.\\

Acest lucru a făcut ca și termenul de fake news să fie definit de ideea fabricării intenționate a unei știri sau informații care nu are o baza reală și care se poate dovedi falsă, scopul fiind derutarea cititorilor de la adevărata informație \cite{FakeNews_Def}.\\

Încrederea revine companiilor de media ce descoperă informația și o împachetează și în final la oamenii ce o pun la dispoziția largă a publicului. Companiile sunt responsabile de informația pe care o prelucrează și de securitatea mediilor prin care aceasta este livrată.\\

Problema intervine atunci când vorbim de mediul online, care în această situație este format atât din platformele web destinate exclusiv publicării știrilor de către trusturi de publicitate cât și din influența mediilor de socializare. Oricine ar trebui să aibă acces la informație însă acest lucru poate să creeaze o portiță de exploatare atacatorilor. 
Dacă securitatea mediilor unde sunt expuse informațiile sunt slabe atunci atacatorii ar putea altera conținutul și ar putea folosi mecanisme de analiză socială pentru a schimba esența informației.\\

Teza de disertație constă în propunerea și crearea unei faze incipiente a unei aplicații concept ce se folosește de tehnologii criptografice puternice precum blockchain și zero-knowledge proofs pentru a adăuga securitate celor două zone de impact (stocare și publicare).\\

\clearpage

Îmbunătățirea directă a securității unei platforme, de exemplu prin criptarea traficului dinspre și spre platformă sau prin implementarea unei sistem de autentificare între forma de publicare și forma de stocare, nu elimină complet posibilitatea ca aceasta să fie atacată. Odată ce se poate trece de mecanismele de securitate și se obține acces la zona de stocare, care în general este sub forma unei baze de date, se pot modifica informațiile din interior. Tehnologiile menționate anterior răspund nevoilor de protecție împotriva unor astfel de atacuri prin modul în care sunt construite și folosite și implicit al proprietăților pe care le dobândesc.  \\

În Capitolul 1 încep să prezint una dintre tehnologiile importante din componența aplicației TrustNews și anume blockchain. Capitolul definește conceptele de bază ale tehnologiei, tipurile de blockchain și diferențele între acestea, structura, motivul pentru care a devenit atât de rapid popular și care sunt situațiile în care el chiar ne poate ajuta.\\

Capitolul 2 merge un pic mai departe și prezintă BigchainDB, o tehnologie dezvoltată pe baza blockchain și care face trecerea ușoară de la clasicele baze de date către un nou concept. Voi vorbi despre ce facilități aduce și de ce este util în anumite aplicații.\\

Capitolul 3 curpinde în detaliu elementele care alcătuiesc un Zero-Knowledge Succint Non-Interactive Argument of Knowledge(zk-SNARK). Prezintă proprietățile și componentele sale în detaliu, prezintă procesele prin care anumite componente se transformă în structuri cu care putem dovedi cunoașterea unei informații fără a o expune.\\

În Capitolul 4 prezint aplicația concept prin care tehnologiile menționate pot fi folosite concomitent. Scopul este de a crea o formă mai sigură și mai eficientă decât un mediu online clasic de unde publicul poate lua informații și unde încrederea este oferită de instrumentele criptografice puternice.\\

Tehnologiile folosite sunt în continuă dezvoltare și astfel pot să apară modificări atât la modul de lucru cu anumite tehnologii cât și la motivația de a le folosi. Aplicația nu își propune să îmbunătățească securitatea actualelor medii online de unde preia informația ci se bazează pe încrederea și reputația companiilor de publicitate că nu publică o informație falsă.\\